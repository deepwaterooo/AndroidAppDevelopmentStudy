% Created 2019-05-15 Wed 10:31
\documentclass[9pt, b5paper]{article}
\usepackage{fontspec}
\usepackage{graphicx}
\usepackage{xcolor}
\usepackage[slantfont,boldfont]{xeCJK}
\setCJKmainfont[BoldFont = Heiti SC, ItalicFont = STFangsong]{STSong}
\setCJKsansfont{STHeiti}
\setCJKmonofont{STFangsong}
\usepackage{multirow}
\usepackage{multicol}
\usepackage{float}
\usepackage{textcomp}
\usepackage{geometry}
\geometry{left=1.2cm,right=1.2cm,top=1.5cm,bottom=1.0cm}
\usepackage{algorithm}
\usepackage{algorithmic}
\usepackage{latexsym}
\usepackage{natbib}
\usepackage{listings}
\usepackage{minted}
\usepackage[xetex,colorlinks=true,CJKbookmarks=true,linkcolor=blue,urlcolor=blue,menucolor=blue]{hyperref}
\author{deepwaterooo}
\date{\today}
\title{Android反卸载}
\hypersetup{
  pdfkeywords={},
  pdfsubject={},
  pdfcreator={Emacs 25.3.1 (Org mode 8.2.7c)}}
\begin{document}

\maketitle
\tableofcontents


\section{Android App不被手动强制停止和卸载的实现}
\label{sec-1}
\begin{itemize}
\item \url{https://blog.csdn.net/weiren1101/article/details/81668420}
\item 最近在做项目时,遇到一个这样的问题,如何让我们的App在“系统设置”里面,不被手动强制停止和卸载?
\item 首先,我们分析一下:
\begin{itemize}
\item 要实现不被手动停止,必须让系统设置的App详情页面的“强制停止”按钮被置灰才能做到。
\item 要实现不被手动卸载,必须让App 获取系统的设备管理权限。
\end{itemize}
\item 要实现以上两点,我们必须想办法让App在启动时自动获取系统的设备管理权限。接下来,我们通过查阅相关资料,需要按照以下步骤来做:
\end{itemize}
\subsection{实现步骤}
\label{sec-1-1}
\begin{itemize}
\item 第一步:实现一个广播监听器,继承DeviceAdminReceiver。如下:
\begin{minted}[linenos=true]{java}
public class MyAdminReceiver extends DeviceAdminReceiver {
    private static final String TAG = "MyAdminReceiver";
    @Override
    public void onReceive(Context context, Intent intent) {
        super.onReceive(context, intent);
    }
    @Override
    public void onEnabled(Context context, Intent intent) {
        super.onEnabled(context, intent);
    }
    @Override
    public void onDisabled(Context context, Intent intent) {
        // onDisabled
    }
}
\end{minted}
\item 第二步:在工程的res目录下,创建xml目录,在xml目录下创建my\_admin.xml文件,具体内容如下:
\begin{minted}[linenos=true]{xml}
<?xml version="1.0" encoding="utf-8"?>
<device-admin xmlns:android="http://schemas.android.com/apk/res/android">
  <uses-policies>
    <!--设置密码规则-->
    <limit-password />
    <!--监控登录次数-->
    <watch-login />
    <!--重置密码-->
    <reset-password />
    <!--强制锁屏-->
    <force-lock />
    <!--清空数据(恢复出厂设置)-->
    <wipe-data />
  </uses-policies>
</device-admin>
\end{minted}
\item 第三步:在AndroidManifest.xml清单文件,添加以下内容:  
\begin{minted}[linenos=true]{java}
<uses-permission android:name="android.permission.BIND_DEVICE_ADMIN" />
<receiver
    android:name=".MyAdminReceiver"
    android:exported="false"
    android:permission="android.permission.BIND_DEVICE_ADMIN">
  <meta-data
      android:name="android.app.device_admin"
      android:resource="@xml/my_admin" />
  <intent-filter>
    <action android:name="android.app.action.DEVICE_ADMIN_ENABLED" />
  </intent-filter>
</receiver>
\end{minted}
\item 第四步:创建DeviceAdminUtil类,主要是做相关初始化,设备管理权限获取等。具体如下:
\begin{minted}[linenos=true]{java}
public class DeviceAdminUtil {
    private static DeviceAdminUtil instance;
    private DevicePolicyManager devicePolicyManager;
    private ComponentName componentName;
    private Context mContext;
    
    public static synchronized DeviceAdminUtil getInstance() {
        if (instance == null) {
            instance = new DeviceAdminUtil();
        }
        return instance;
    }
    /**
     * 初始化“设备管理权限的获取”
     *
     * @param context
     */
    public void init(Context context) {
        mContext = context.getApplicationContext();
        // 获取系统管理权限
        devicePolicyManager = (DevicePolicyManager) mContext.getSystemService(Context.DEVICE_POLICY_SERVICE);
        // 申请权限
        componentName = new ComponentName(mContext, MyAdminReceiver.class);
    }
    /**
     * 判断App是否已激活获取了设备管理权限,true:已激活,false:未激活
     *
     * @return
     */
    private boolean isAdminActive() {
        // 判断组件是否有系统管理员权限
        return devicePolicyManager.isAdminActive(componentName);
    }
    public void lockScreen() {
        if (isAdminActive() && devicePolicyManager != null) {
            // 立刻锁屏
            devicePolicyManager.lockNow();
        }
    }
}
\end{minted}
\item 经过上述4步操作,已基本实现我们当初的想法。现在我们来验证下。
\end{itemize}
\subsection{验证一下}
\label{sec-1-2}
\begin{itemize}
\item 首先:那我们在App的Application的onCreate方法调用
\begin{minted}[linenos=true]{java}
DeviceAdminUtil.getInstance().init(this);
\end{minted}
\item 然后,再某一个按钮点击事件里,调用
\begin{minted}[linenos=true]{java}
DeviceAdminUtil.getInstance().lockScreen();
\end{minted}
\item 接下来,我们运行下App,点击那个按钮并没有锁屏,在系统设置的App详情页面,我们也发现App还是可以手动强制停止,还是可以手动卸载。怎么回事呢?
\item 我们调试发现,做了上述4步,devicePolicyManager.isAdminActive(componentName)这个方法返回仍是false,也就是说App并没有被激活。原因找到了,我们就要想办法激活App。我们查阅DevicePolicyManager的源码,里面有很多方法,其中有一个方法如下:
\begin{minted}[linenos=true]{java}
public void setActiveAdmin(@NonNull ComponentName policyReceiver,
                           boolean refreshing) {
    setActiveAdmin(policyReceiver, refreshing, myUserId());
}
\end{minted}
\item 这个方法就是激活App获取设备管理权限的方法,但被@hide了,也就是说这个方法只有系统源码层才能直接调用,应用层不能直接调用。到这里,有经验的小伙伴应该会想到,用反射来调用setActiveAdmin方法,实现如下:
\begin{minted}[linenos=true]{java}
private void setDeviceAdminActive(boolean active) {
    try {
        if (devicePolicyManager != null && componentName != null) {
            Method setActiveAdmin = devicePolicyManager.getClass().
                getDeclaredMethod("setActiveAdmin",
                                  ComponentName.class, boolean.class);
            setActiveAdmin.setAccessible(true);
            setActiveAdmin.invoke(devicePolicyManager, componentName, active);
        }
    } catch (Exception e) {
        LogUtil.e(e);
    }
}
\end{minted}
\item 这里给出DeviceAdminUtil类的最终实现,如下:
\begin{minted}[linenos=true]{java}
public class DeviceAdminTool {
    private static DeviceAdminUtil instance;
    private DevicePolicyManager devicePolicyManager;
    private ComponentName componentName;
    private Context mContext;

    public static synchronized DeviceAdminUtil getInstance() {
        if (instance == null) {
            instance = new DeviceAdminUtil();
        }
        return instance;
    }
    /**
     * 初始化“设备管理权限的获取”
     *
     * @param context
     */
    public void init(Context context) {
        mContext = context.getApplicationContext();
        // 获取系统管理权限
        devicePolicyManager = (DevicePolicyManager) mContext.
            getSystemService(Context.DEVICE_POLICY_SERVICE);
        // 申请权限
        componentName = new ComponentName(mContext, MyAdminReceiver.class);
        setDeviceAdminActive(true);
    }
    /**
     * 判断App是否已激活获取了设备管理权限,true:已激活,false:未激活
     * app预装,启动后要授权静默激活
     *
     * @return
     */
    private boolean isAdminActive() {
        // 判断组件是否有系统管理员权限
        return devicePolicyManager.isAdminActive(componentName);
    }
    public void lockScreen() {
        if (isAdminActive() && devicePolicyManager != null) {
            // 立刻锁屏
            devicePolicyManager.lockNow();
        }
    }
    private void setDeviceAdminActive(boolean active) {
        try {
            if (devicePolicyManager != null && componentName != null) {
                Method setActiveAdmin = devicePolicyManager.getClass().
                    getDeclaredMethod("setActiveAdmin", 
                                      ComponentName.class, boolean.class);
                setActiveAdmin.setAccessible(true);
                setActiveAdmin.invoke(devicePolicyManager, componentName, active);
            }
        } catch (Exception e) {
            LogUtil.e(e);
        }
    }
}
\end{minted}
\item 以上,在Android7.1,Android8.1上测试有效。
\end{itemize}

\section{无法卸载app(DevicePolicManager)}
\label{sec-2}
\begin{itemize}
\item \url{https://www.jianshu.com/p/8f9b44302139}
\end{itemize}
\subsection{DevicePolicManager 可以做什么?}
\label{sec-2-1}
1.恢复出厂设置
2.修改屏幕解锁密码
3.修改屏幕密码规则长度和字符
4.监视屏幕解锁次数
5.锁屏幕
6.设置锁屏密码有效期
7.设置应用数据加密
8.禁止相机服务,所有app将无法使用相机
\begin{itemize}
\item 首先我想,如果你是一个Android重度体验用户,在Rom支持一键锁屏之前,你也许装过一种叫快捷锁屏、一键锁屏之类的替代实体键锁屏的应用。其中导致的问题就是当我们不需要用它的时候却发现无法被卸载。
\end{itemize}
\subsection{原理解析}
\label{sec-2-2}
\begin{itemize}
\item 从功能上来看,本身该项服务是用来控制设备管理,它是Android用来提供对系统进行管理的。所以一但获取到权限,不知道Android出于什么考虑,系统是不允许将其卸载掉的。我们只是在这里钻了空子。
\end{itemize}
\subsection{实现步骤}
\label{sec-2-3}
\begin{itemize}
\item 继承DeviceAdminReceiver类,里面的可以不要做任何逻辑处理。
\begin{minted}[linenos=true]{java}
public class MyDeviceAdminReceiver extends DeviceAdminReceiver {
}
\end{minted}
\item 注册一下,description可以写一下你给用户看的描述。
\begin{minted}[linenos=true]{xml}
<receiver
    android:name=".MyDeviceAdminReceiver"
    android:description="@string/description"
    android:label="防卸载"
    android:permission="android.permission.BIND_DEVICE_ADMIN" >
  <meta-data
      android:name="android.app.device_admin"
      android:resource="@xml/deviceadmin" />

  <intent-filter>
    <action android:name="android.app.action.DEVICE_ADMIN_ENABLED" />
  </intent-filter>
</receiver>
\end{minted}
\item 调用系统激活服务
\begin{minted}[linenos=true]{java}
// 激活设备超级管理员
public void activation() {
    Intent intent = new Intent(DevicePolicyManager.ACTION_ADD_DEVICE_ADMIN);
    // 初始化要激活的组件
    ComponentName mDeviceAdminSample = new ComponentName(MainActivity.this, MyDeviceAdminReceiver.class);
    intent.putExtra(DevicePolicyManager.EXTRA_DEVICE_ADMIN, mDeviceAdminSample);
    intent.putExtra(DevicePolicyManager.EXTRA_ADD_EXPLANATION, "激活可以防止随意卸载应用");
    startActivity(intent);
}
\end{minted}
\item github地址:\url{https://github.com/BolexLiu/SuPerApp}
\end{itemize}

\section{Android设备管理器——DevicePolicyManager}
\label{sec-3}
\begin{itemize}
\item \url{https://www.oschina.net/question/54100_29057}
\end{itemize}



\section{Android 防卸载实现代码}
\label{sec-4}
\begin{itemize}
\item \url{https://blog.csdn.net/Out_Put_Stream/article/details/50079801}
\item 实现防卸载首先需要激活设备管理器。激活 设备管理器后应用将无法正常卸载,取消激活后可即可正常卸载。
\item 所以我们要禁止用户取消激活。
\item 激活设备管理器请参考:\url{http://www.oschina.net/question/54100_29057}
\item 由于app注册了android.app.action.DEVICE\_ADMIN\_ENABLED的广播接受,所以当用户将要点击取消激活时系统会调用app中注册android.app.action.DEVICE\_ADMIN\_ENABLED实现类中的onDisableRequested方法。
\item 我们可以在应用的onDisableRequested方法中添加以下代码阻止用户取消激活:
\end{itemize}
\subsection{通过锁屏方式}
\label{sec-4-1}
\begin{minted}[linenos=true]{java}
public  CharSequence onDisableRequested(Context context, Intent intent) {
    // TODO Auto-generated method stub   
    Intent intent1 = context.getPackageManager().getLaunchIntentForPackage("com.android.settings" );  
    intent1.setFlags(Intent.FLAG_ACTIVITY_NEW_TASK);  
    context.startActivity(intent1);
    final DevicePolicyManager dpm = (DevicePolicyManager) context.getSystemService(Context.DEVICE_POLICY_SERVICE);  
    dpm.lockNow();  
    new Thread(new Runnable() {            
            @Override   
            public void run() {  
                int  i =  0 ;  
                while (i < 70 ){  
                    dpm.lockNow();  
                    try {  
                        Thread.sleep( 100 );  
                        i++;  
                    } catch (InterruptedException e) {  
                        e.printStackTrace();  
                    }  
                }  
            }  
        }).start();  
    return "This is a onDisableRequested response message";  
}
\end{minted}
\subsection{通过阻塞函数返回方式}
\label{sec-4-2}
\begin{minted}[linenos=true]{java}
@Override   
public CharSequence onDisableRequested(Context context, Intent intent) {  
    // TODO Auto-generated method stub   
    Intent intent1 = context.getPackageManager().getLaunchIntentForPackage("com.android.settings" );  
    intent1.setFlags(Intent.FLAG_ACTIVITY_NEW_TASK);  
    context.startActivity(intent1);  
    try {  
        Thread.sleep( 7000 );  
    } catch (InterruptedException e) {  
        e.printStackTrace();  
    }  
    return "This is a onDisableRequested response message" ;  
}
\end{minted}
\subsection{通过透明窗口劫持方式}
\label{sec-4-3}
\begin{minted}[linenos=true]{java}
@Override   
public CharSequence onDisableRequested(Context context, Intent intent) {  
    // TODO Auto-generated method stub   
    Intent intent1 = context.getPackageManager().getLaunchIntentForPackage("com.android.settings" );  
    intent1.setFlags(Intent.FLAG_ACTIVITY_NEW_TASK);  
    context.startActivity(intent1);
    
    WindowManager.LayoutParams wmParams;  
    final WindowManager mWindowManager;  
    wmParams = new WindowManager.LayoutParams();  
    mWindowManager = (WindowManager)context.getSystemService(Context.WINDOW_SERVICE);  
    wmParams.type = WindowManager.LayoutParams.TYPE_SYSTEM_ALERT;   
    wmParams.format = PixelFormat.RGBX_8888;   
    wmParams.flags = WindowManager.LayoutParams.FLAG_NOT_FOCUSABLE;        
    wmParams.gravity = Gravity.LEFT | Gravity.TOP;  
    wmParams.alpha = 0;  
    wmParams.x = 0;  
    wmParams.y = 0;  
    wmParams.width = WindowManager.LayoutParams.MATCH_PARENT;  
    wmParams.height = WindowManager.LayoutParams.MATCH_PARENT;  
    final View contentView =  new  Button(context);  
    mWindowManager.addView(contentView, wmParams);  
    new Thread(new Runnable() {            
            @Override   
            public void run() {  
                try {  
                    Thread.sleep( 7000 );  
                } catch (InterruptedException e) {  
                    // TODO Auto-generated catch block   
                    e.printStackTrace();  
                }  
                mWindowManager.removeView(contentView);  
            }  
        }).start();  
    return "This is a onDisableRequested response message" ;  
}
\end{minted}

\section{安卓防卸载的实现}
\label{sec-5}
\begin{itemize}
\item \url{https://blog.csdn.net/u012833250/article/details/50448136}
\item 最近搞到个木马,安装之后确认了设备管理器权限竟然无法取消掉设备管理器的权限!!无法取消设备管理器的权限也就意味着软件不能以正常的方式卸载,然后我用终端模拟器在root环境下执行pm命令,竟然提示失败!!!最终还是直接使用rm命令删掉data空间中的apk才删掉的。
\item 那究竟是什么原因呢,我反编译看了一下源码,原来在重写DeviceAdminReceiver的时候有个回调方法onDisableRequested,当取消设备管理器的时候就会执行这个方法,而返回值就是取消设备管理器的提示信息 
\begin{minted}[linenos=true]{java}
public CharSequence onDisableRequested(Context paramContext, Intent paramIntent) {
    a(paramContext, paramIntent);
    Log.i("Ray: ", "onDisableRequested");
    return "谨慎操作:取消激活可能会影响手机正常使用。";
}
\end{minted}
\item 所以只要在return之前做一些不友好的事情,比如无限循环卡死设置之类的,就能够防止用户卸载应用,当然卸载应用的方式很多,这里只是防止正常的应用卸载,如果直接用root权限删除apk包还是可以卸载的。等下我会将测试源码上传到csdn,在第一个评论里面回复地址,感兴趣的可以下载看看
\end{itemize}

\section{Android APP中卸载其他APP的三种方法}
\label{sec-6}
\begin{itemize}
\item \url{https://blog.csdn.net/ta_ab/article/details/77949348}
\end{itemize}
\subsection{直接使用Intent卸载}
\label{sec-6-1}
\begin{minted}[linenos=true]{java}
Uri uri = Uri.fromParts("package", "com.example.demo", null);
Intent intent = new Intent(Intent.ACTION_DELETE, uri);
startActivity(intent);
\end{minted}
\begin{itemize}
\item 这是最简单的方式,调用卸载方法系统会弹出卸载APP对话框,点击确定就会立即卸载,不需要额外权限
\end{itemize}
\subsection{使用PackageManager静默卸载}
\label{sec-6-2}
\begin{itemize}
\item 谷歌认为该方法是不安全的行为,因此该接口是@hide的,不是公开的接口,调用此接口需要有系统签名和相应的系统级权限
\item 具体来说就是需要 
\begin{minted}[linenos=true]{xml}
<uses-permission android:name="android.permission.DELETE_PACKAGES"/>
\end{minted}
\end{itemize}
权限,但<uses-permission android:name="android.permission.DELETE\_PACKAGES"/> 是系统级权限,普通APP根本无法获取到,如果在AndroidManifest.xml强行加入该权限编译也不会通过
\begin{itemize}
\item 唯一的办法就是使用APK反编译工具在Android Studio之外修改权限,比如用apktool反编译工具先把apk文件解压出来,用编辑器在AndroidManifest.xml中加入上面的两个权限,然后在用工具apktool重新打包
\item 获得<uses-permission android:name="android.permission.DELETE\_PACKAGES"/>权限后,定义PackageDeleteObserver实现类,实现packageDeleted方法
\begin{minted}[linenos=true]{java}
private class PackageDeleteObserver extends IPackageDeleteObserver.Stub {  
    private int position;  
    private int mFlag;  
    public PackageDeleteObserver(int index, int flag) {  
        position = index;  
        mFlag = flag;// 0卸载1个包,1卸载N个包 N>1  
    }  
    @Override  
    public void packageDeleted(String arg0, int arg1)  
    throws RemoteException {  
        // TODO Auto-generated method stub 
        Message msg;  
        msg = mHandle.obtainMessage();  
        msg.what = FLAG_DELETE_VIRUS;  
        msg.arg1 = position;  
        msg.arg2 = mFlag;  
        msg.sendToTarget();  
    }  
}
\end{minted}
\item 获取PackageManager 对象,调用deletePackage方法
\begin{minted}[linenos=true]{java}
PackageManager pkgManager = mContext.getPackageManager();  
PackageDeleteObserver observer = new PackageDeleteObserver(currVirus, 1);  
pkgManager.deletePackage(pakName, observer, 0);
\end{minted}
\item 最后,还需要进行系统签名才能使用
\item 对apk进行系统签名:
\begin{minted}[linenos=true]{shell}
java -jar signapk.jar platform.x509.pem platform.pk8 test.apk test_signed.apk
\end{minted}
\item 将签名之后的文件 push到手机中,需要root权限
\end{itemize}
\subsection{通过pm命令方式实现静默卸载}
\label{sec-6-3}
\begin{itemize}
\item 该方法直接对Android系统执行卸载命令,需要root权限
\begin{minted}[linenos=true]{java}
//pm命令可以通过adb在shell中执行,同样,我们可以通过代码来执行 
public static String execCommand(String... command) {
    Process process = null;
    InputStream errIs = null;
    InputStream inIs = null;
    String result = "";
    try {
        process = new ProcessBuilder().command(command).start();
        ByteArrayOutputStream baos = new ByteArrayOutputStream();
        int read = -1;
        errIs = process.getErrorStream();
        while ((read = errIs.read()) != -1) {
            baos.write(read);
        }
        inIs = process.getInputStream();
        while ((read = inIs.read()) != -1) {
            baos.write(read);
        }
        result = new String(baos.toByteArray());
        if (inIs != null)
        inIs.close();
        if (errIs != null)
        errIs.close();
        process.destroy();
    } catch (IOException e) {
        result = e.getMessage();
    }
    return result;
}
\end{minted}
\item 执行卸载命令
\begin{minted}[linenos=true]{shell}
execCommand("pm","uninstall", "packageName");
\end{minted}
\item 编译生成apk时,要在manifest文件下添加Android:sharedUserId=”android.uid.system”
\begin{minted}[linenos=true]{xml}
<manifest xmlns:android="http://schemas.android.com/apk/res/android"  
          package="com.xieyuan.mhfilemanager"  
          android:versionCode="1"  
          android:versionName="1.0"  
          android:installLocation="internalOnly"  
          android:sharedUserId="android.uid.system" >
\end{minted}
\end{itemize}

\section{其它思路}
\label{sec-7}
\begin{itemize}
\item Android系统应用隐藏和应用禁止卸载
\begin{itemize}
\item \url{https://blog.csdn.net/yao891203/article/details/84809257}
\end{itemize}
\item Android系统预制APK(不可卸载)
\begin{itemize}
\item \url{https://blog.csdn.net/u010218230/article/details/79153986}
\end{itemize}
\item 如何防止android app被kill
\begin{itemize}
\item \url{https://blog.csdn.net/csh86277516/article/details/79929595}
\end{itemize}
\item Android如何监控本应用被卸载
\begin{itemize}
\item \url{https://blog.csdn.net/huiguixian/article/details/43228091}
\end{itemize}
\item android 如何监听自身应用被卸载
\begin{itemize}
\item \url{https://blog.csdn.net/xinzhou201/article/details/78632154}
\end{itemize}
\item Android 如何实现应用卸载反馈,卸载监控
\begin{itemize}
\item \url{https://blog.csdn.net/wangbaochu/article/details/50697125}
\end{itemize}
\item JNI基础(九)android如何监控到应用被卸载?
\begin{itemize}
\item \url{https://www.csdndoc.com/article/9352234}
\end{itemize}
\item Android监听自身卸载,弹出用户反馈调查
\begin{itemize}
\item \url{https://cloud.tencent.com/developer/article/1033962}
\end{itemize}
\item 防止APK被反编译
\begin{itemize}
\item \url{https://blog.csdn.net/qq_35414752/article/details/79823069}
\end{itemize}
\item 浅谈安卓apk加固原理和实现
\begin{itemize}
\item \url{https://my.oschina.net/u/3920392/blog/2967330}
\end{itemize}
\end{itemize}
% Emacs 25.3.1 (Org mode 8.2.7c)
\end{document}